\documentclass{beamer}

\usetheme{Warsaw}
\usefonttheme{professionalfonts}
%\logo{\includegraphics[height=1cm]{./images/logo.png}}
\beamertemplatenavigationsymbolsempty
\setbeamertemplate{caption}{\raggedright\insertcaption\par}
\usepackage[T1]{fontenc}
\usepackage[utf8]{inputenc}
\usepackage[spanish]{babel}
\usepackage{float}
\usepackage{framed}
\usepackage{geometry}
\usepackage{tikz}
\usepackage{amssymb}
\usepackage{changepage}
\usepackage{xcolor}
\usepackage{minted}
\usepackage{fancyvrb}
\usepackage{relsize}
\usepackage{wasysym}
\usepackage{textcomp}
\usepackage{hyperref}
\usepackage{verbatim}
\usepackage{graphicx}
\usepackage{subcaption}
\definecolor{LightGray}{gray}{0.9}

%\setbeamertemplate{footline}[frame number]

\begin{document}

\title[Presentación del curso]{Teoría de la Computación \\ Semestre 2-2025}
\author[Teoría de la Computación]{\textbf{Profesores:}\\ Consuelo Ramírez \\
  Cristóbal Loyola}
% \date{28 de marzo de 2025}
\date{}

\titlegraphic{
  \vspace{-2cm}
  \begin{figure}
    \centering
    \begin{subfigure}{0.24\textwidth}
        \centering
        \includegraphics[width=\linewidth]{images/ada_lovelace.jpg}
    \end{subfigure}
    \hfill
    \begin{subfigure}{0.24\textwidth}
        \centering
        \includegraphics[width=\linewidth]{images/frege.jpg}
    \end{subfigure}
    \hfill
    \begin{subfigure}{0.24\textwidth}
        \centering
        \includegraphics[width=\linewidth]{images/turing.jpg}
    \end{subfigure}
    \hfill
    \begin{subfigure}{0.24\textwidth}
        \centering
        \includegraphics[width=\linewidth]{images/grace_hopper.jpg}
    \end{subfigure}
  \end{figure}
}

\frame{\titlepage}

\begin{frame}{Resultados de aprendizaje general}
  Resolver problemas en contextos variados aplicando conceptos de la Teoría de
  la Computación por medio de la representación y reconocimiento de
  \textbf{lenguajes formales, modelos de lógica clásica/difusa, computabilidad y
    complejidad computacional}; desarrollando la capacidad de adaptación a las
  condiciones de cada problema y los recursos disponibles, resguardando el
  compromiso con la ética y el trabajo bien realizado.
\end{frame}


\begin{frame}{Resumen del curso}
  \begin{itemize}[<+->]
    \item Desarrollar herramientas matemáticas para entender algoritmos y
          procesos computacionales.
    \item Analizar lenguajes formales y gramáticas, lo que es fundamental para
          entender el funcionamiento de lenguajes de programación y
          compiladores.
    \item Explorar qué significa que un problema sea computable, a través de
          modelos como autómatas y máquinas de Turing. Entender las
          limitaciones de estos modelos.
    \item Analizar los recursos (tiempo y espacio) requeridos para computar y
          clasificar problemas basado en su complejidad computacional.
  \end{itemize}
\end{frame}


\begin{frame}{Unidades}
  \begin{enumerate}[<+->]
    \item Lógica $\rightarrow$ (6 clases + 3 ayudantías)
    \item Lenguajes Regulares $\rightarrow$ (8 clases + 3 ayudantías)
    \item Lenguajes Independientes del Contexto $\rightarrow$ (6 clases +
          1 ayudantía)
    \item Computabilidad y Complejidad $\rightarrow$ (8 clases + 3 ayudantías)
  \end{enumerate}

  % \begin{figure}
  %   \centering
  %   \begin{subfigure}{0.24\textwidth}
  %       \centering
  %       \includegraphics[width=\linewidth]{images/ada_lovelace.jpg}
  %   \end{subfigure}
  %   \hfill
  %   \begin{subfigure}{0.24\textwidth}
  %       \centering
  %       \includegraphics[width=\linewidth]{images/frege.jpg}
  %   \end{subfigure}
  %   \hfill
  %   \begin{subfigure}{0.24\textwidth}
  %       \centering
  %       \includegraphics[width=\linewidth]{images/turing.jpg}
  %   \end{subfigure}
  %   \hfill
  %   \begin{subfigure}{0.24\textwidth}
  %       \centering
  %       \includegraphics[width=\linewidth]{images/grace_hopper.jpg}
  %   \end{subfigure}
  % \end{figure}
\end{frame}


\begin{frame}{Evaluaciones y asistencia}
  \begin{columns}
    \begin{column}{0.6\textwidth}
      \begin{itemize}
        \item \textbf{PEP 1} (15\%): 12 de septiembre
        \item \textbf{PEP 2} (30\%): 17 de octubre
        \item \textbf{PEP 3} (25\%): 7 de noviembre
        \item \textbf{PEP 4} (30\%): 2 de diciembre
      \end{itemize}
    \end{column}

    \begin{column}{0.4\textwidth}
      \begin{itemize}
        \item \textbf{PR}: 9 de diciembre
        \item \textbf{PDR}: 16 de diciembre
      \end{itemize}
    \end{column}
  \end{columns}

  \begin{itemize}[<+->]
    \vspace{0.3cm}
    \item El curso tiene 5 SCT y TEL: 4-2-0
    \item La condición para rendir la PDR es tener un promedio de PEPs mayor o
          igual que 3,0.
    \item La PDR reemplaza la calificación de la PEP que más desfavorece al
          promedio.
    \item La asistencia a clases es obligatoria en un 75\% como requisito de
          aprobación.
  \end{itemize}
\end{frame}


\begin{frame}{Bibliografía}
  \begin{itemize}
    \item Aho, A., Lam, M., Sethi, R. y Ullman, J. (2008). Compiladores:
          Principios, técnicas y herramientas (2a ed.). Pearson Addison Wesley.
    \item Garrido, M. (2001). Lógica simbólica (4a ed.). Tecnos.
    \item Hopcroft, J., Motwani, R. y Ullman, J. (2002). Introducción a la
          Teoría de Autómatas, Lenguajes y Computación (2a ed.). Addison Wesley.
    \item Lewis, H. \& Papadimitriou, C. (1998). Elements of the Theory of
          Computation (2nd ed.). Upper Saddle River: Prentice-Hall.
    \item Sipser, M. (2013). Introduction to the Theory of Computation (3a ed.).
          Cengage Learning.
    \item Zhang, H., Zhang, J. (2024). Logic in Computer Science (2024th ed.).
          Springer.
  \end{itemize}
\end{frame}

\end{document}
