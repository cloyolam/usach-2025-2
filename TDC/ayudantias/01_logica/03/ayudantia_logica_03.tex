\documentclass{article}
\usepackage{amsmath}
\usepackage{amssymb}
\usepackage[spanish]{babel}
\usepackage{color}
\usepackage{colortbl}
\usepackage{float}
\usepackage[T1]{fontenc}
\usepackage[rmargin=3cm,lmargin=3cm,tmargin=3cm,bmargin=4cm]{geometry}
\usepackage[utf8]{inputenc}
\usepackage{latexsym}
\usepackage{multirow}
\usepackage[spanish]{syllogism}
\usepackage{tikz}
\usepackage{tikz-qtree}
\usepackage{upgreek}

\clubpenalty=10000
\widowpenalty=10000

\begin{document}

\title{Ayudantía Unidad 1: Lógica (parte 3)}
\author{Teoría de la Computación 2-2025}
\date{}

\maketitle

\section{Lógica de primer orden: UMG}

Determine, si es posible, el unificador de máxima generalidad, indicando cada uno de los pasos para su obtención:

\begin{itemize}
  \item $t1 = q(f(a),\ g(b, Y),\ m(X, f(Z)))$ y
        $t2 = q(X,\ g(b, c),\ m(f(a), Z))$
\end{itemize}



\end{document}
