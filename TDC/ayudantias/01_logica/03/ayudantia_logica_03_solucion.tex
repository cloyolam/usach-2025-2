\documentclass{article}
\usepackage{amsmath}
\usepackage{amssymb}
\usepackage[spanish]{babel}
\usepackage{color}
\usepackage{colortbl}
\usepackage{float}
\usepackage[T1]{fontenc}
\usepackage[rmargin=3cm,lmargin=3cm,tmargin=3cm,bmargin=4cm]{geometry}
\usepackage[utf8]{inputenc}
\usepackage{latexsym}
\usepackage{multirow}
\usepackage[spanish]{syllogism}
\usepackage{tikz}
\usepackage{tikz-qtree}
\usepackage{upgreek}

\clubpenalty=10000
\widowpenalty=10000

\begin{document}

\title{Ayudantía Unidad 1: Lógica (parte 3)}
\author{Teoría de la Computación 2-2025}
\date{}

\maketitle

\section{Lógica de primer orden: UMG}

Determine, si es posible, el unificador de máxima generalidad, indicando cada uno de los pasos para su obtención:

\subsection{UMG 1}

$$E = f(x_{1},\ x_{3},\ x_{2})$$
$$F = f(g(x_{2}),\ j(x_{4}),\ h(x_{3}, a))$$

Aplicando el algoritmo de Robinson:

\begin{itemize}
  \item $k = 0$
  \begin{align*}
    &\sigma_{0} = \{\}\\
    &E_{0} = \sigma_{0}(E) = f(x_{1},\ x_{3},\ x_{2})\\
    &F_{0} = \sigma_{0}(F) = f(g(x_{2}),\ j(x_{4}),\ h(x_{3}, a))
  \end{align*}


  \item $k = 1$: par de discordancia $(x_{1},\ g(x_{2}))$
  \begin{align*}
    &\sigma_{1} = \{x_{1} / g(x_{2})\}\\
    &E_{1} = \sigma_{1}(E_{0}) = f(g(x_{2}),\ x_{3},\ x_{2})\\
    &F_{1} = \sigma_{1}(F_{0}) = f(g(x_{2}),\ j(x_{4}),\ h(x_{3}, a))
  \end{align*}


  \item $k = 2$: par de discordancia $(x_{3},\ j(x_{4}))$
  \begin{align*}
    &\sigma_{2} = \{x_{3}/j(x_{4})\}\\
    &E_{2} = \sigma_{2}(E_{1}) = f(g(x_{2}),\ j(x_{4}),\ x_{2})\\
    &F_{2} = \sigma_{2}(F_{1}) = f(g(x_{2}),\ j(x_{4}),\ h(j(x_{4}), a))
  \end{align*}


  \item $k = 3$: par de discordancia $(x_{2},\ h(j(x_{4}), a))$
  \begin{align*}
    &\sigma_{3} = \{x_{2}/h(j(x_{4}, a))\}\\
    &E_{3} = \sigma_{3}(E_{2}) = f(g(h(j(x_{4}, a))),\ j(x_{4}),\ h(j(x_{4}, a)))\\
    &F_{3} = \sigma_{3}(F_{2}) = f(g(h(j(x_{4}, a))),\ j(x_{4}),\ h(j(x_{4}), a))
  \end{align*}
\end{itemize}

Dado que $E_{3} = F_{3}$, logramos unificar las expresiones $E$ y $F$ originales
con el UMG:
$$\sigma = \sigma_{3} \circ \sigma_{2} \circ \sigma_{1} = \{x_{1} / g(h(j(x_{4}, a))),\ x_{3}/j(x_{4}),\ x_{2}/h(j(x_{4}, a))\}$$


\subsection{UMG 2}
$$E = q(f(a),\ g(b, Y),\ m(X, f(Z)))$$
$$F = q(X,\ g(b, c),\ m(f(a), Z))$$


\section{Lógica de primer orden: resolución}

\subsection{Resolución 1}
Demuestre usando resolución la validez de:
$$\forall x [P(x) \rightarrow Q(x)] \vDash \forall y [\neg Q(y) \rightarrow \neg P(y)]$$

\subsection{Resolución 2}
Exprese los siguientes enunciados como fórmulas de lógica de primer orden:
\begin{enumerate}
  \item Todo dragón es feliz si todos sus hijos pueden volar.
  \item Los dragones verdes pueden volar.
  \item Un dragón es verde si es hijo de al menos un dragón verde.
\end{enumerate}

Demuestre usando resolución que la conjunción de estos 3 enunciados implica lo
siguiente: todos los dragones verdes son felices.

\end{document}
