\documentclass{article}
\usepackage{amsmath}
\usepackage{amssymb}
\usepackage[spanish]{babel}
\usepackage{color}
\usepackage{colortbl}
\usepackage{float}
\usepackage[T1]{fontenc}
\usepackage[rmargin=3cm,lmargin=3cm,tmargin=3cm,bmargin=4cm]{geometry}
\usepackage[utf8]{inputenc}
\usepackage{latexsym}
\usepackage{multirow}
\usepackage[spanish]{syllogism}
\usepackage{tikz}
\usepackage{tikz-qtree}
\usepackage{upgreek}

\clubpenalty=10000
\widowpenalty=10000

\begin{document}

\title{Ayudantía Unidad 1: Lógica (parte 2)}
\author{Teoría de la Computación 2-2025}
\date{}

\maketitle

\section{Lógica proposicional: Resolución}
Utilice el método de resolución para determinar si los siguientes enunciados son
correctos:

\begin{enumerate}
  \item $\Sigma \vDash C$ para el conjunto de premisas:

$$\Sigma = \{A \rightarrow (B \vee C),\ A,\ \neg B\}$$

  \item Si consumo frutas, entonces tengo energía. Puedo consumir frutas o
        comida chatarra. Si consumo comida chatarra, entonces me enfermo. Por lo
        tanto, tengo energía o estoy enfermo.

  \item $\Sigma \vDash p \rightarrow (\neg r \land \neg s)$ para el conjunto de
        premisas:

$$\Sigma = \{p \rightarrow \neg q,\ \neg q \rightarrow (\neg r \land s), \ t,\ t\rightarrow q\}$$
\end{enumerate}


\section{Lógica de primer orden: formas normales}

\begin{enumerate}
  \item Obtenga la forma normal prenexa de:
  $$\neg[\forall x \exists y\ M(x, y, z) \rightarrow \exists x (\neg \forall y\ G(y, w) \rightarrow H(x))]$$

  \item Obtenga la forma normal de Skolem de:
  $$\neg \forall x \exists r \forall y \exists z \exists w [(\neg S(x, z) \land P(b, y)) \vee (\neg P(x, z) \land S(w, r))]$$


\end{enumerate}


\end{document}
