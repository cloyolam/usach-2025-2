\documentclass{article}
\usepackage{amsmath}
\usepackage{amssymb}
\usepackage[spanish]{babel}
\usepackage{color}
\usepackage{colortbl}
\usepackage{float}
\usepackage[T1]{fontenc}
\usepackage[rmargin=3cm,lmargin=3cm,tmargin=3cm,bmargin=4cm]{geometry}
\usepackage[utf8]{inputenc}
\usepackage{latexsym}
\usepackage{multirow}
\usepackage[spanish]{syllogism}
\usepackage{tikz}
\usepackage{tikz-qtree}

\clubpenalty=10000
\widowpenalty=10000

\begin{document}

\title{Ayudantía Unidad 1: Lógica (parte 1)}
\author{Teoría de la Computación 2-2025}
\date{}

\maketitle

\section{Tablas de verdad}

\subsection{Reglas generales}

\begin{itemize}
  \item La cantidad de filas que tendrá la tabla es $2^n$, siendo $n$ el número
        de proposiciones simples que hay en la fórmula.
  \item Para establecer todas las combinaciones posibles de valores de verdad
        asignaremos a la primera mitad de las filas de $p$ el valor 0, y a la
        otra mitad, el valor 1. Luego, a $q$ le asignaremos 0 a la mitad de la
        mitad de $p$, luego 1 a la otra mitad de la mitad y así sucesivamente.
        \textbf{Ejemplo:}
  \begin{table}[H]
    \centering
    \begin{tabular}{c|c}
      $p$ & $q$ \\
      \hline
      \hline
      0 & 0 \\
      0 & 1 \\
      1 & 0 \\
      1 & 1 \\
    \end{tabular}
  \end{table}
  \item Para cada fórmula, indique si corresponde a una tautología, a una
        contingencia o a una contradicción.
\end{itemize}

\begin{enumerate}
  \item $\sim (p \vee \sim p) \land (q \land(\sim p \vee r))$

  \item $\sim (p \land \sim q) \leftrightarrow (p \rightarrow q)$

  \item
        $(p \vee q) \land ((r \leftrightarrow \sim s) \rightarrow (\sim r \land
        p))$
\end{enumerate}


\section{Deducción natural}
Demuestre los siguientes secuentes utilizando deducción natural. En cada paso,
debe indicar exactamente cuál fue la regla utilizada y sobre qué fórmulas se
aplicó.

\begin{enumerate}

  \item
        $p \land (q\rightarrow (p \rightarrow s)), p \rightarrow (q \land r) \vdash p \rightarrow s$

  \item
        $p \rightarrow (q \rightarrow r \vee s), p, r \rightarrow t, s \rightarrow t, q\land m \vdash t$

  \item $p \land (\sim q \rightarrow \sim p) \vdash (q \land p) \vee \sim p$
        \quad (PEP 1 2025-1)

  \item
        $((p \land q) \rightarrow (r \land s)) \land ((r \land s) \rightarrow (p \land q)), t \land (t \rightarrow s) \vdash r \rightarrow p$
        \quad (PEP 1 2025-1)


%  \item $(p \vee q) \vee (\sim r \rightarrow s), q \rightarrow r, \sim s \vdash p \vee r$ %\quad \textbf{(Ayudantía)}

\end{enumerate}


\end{document}
