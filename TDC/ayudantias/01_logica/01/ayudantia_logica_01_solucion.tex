\documentclass{article}
\usepackage{amsmath}
\usepackage{amssymb}
\usepackage[spanish]{babel}
\usepackage{color}
\usepackage{colortbl}
\usepackage{float}
\usepackage[T1]{fontenc}
\usepackage[rmargin=3cm,lmargin=3cm,tmargin=3cm,bmargin=4cm]{geometry}
\usepackage[utf8]{inputenc}
\usepackage{latexsym}
\usepackage{multirow}
\usepackage[spanish]{syllogism}
\usepackage{tikz}
\usepackage{tikz-qtree}

\clubpenalty=10000
\widowpenalty=10000

\begin{document}

\title{Ayudantía Unidad 1: Lógica (parte 1)}
\author{Teoría de la Computación 2-2025}
\date{}

\maketitle

\section{Tablas de verdad}

\subsection{Reglas generales}

\begin{itemize}
  \item La cantidad de filas que tendrá la tabla es $2^n$, siendo $n$ el número
        de proposiciones simples que hay en la fórmula.
  \item Para establecer todas las combinaciones posibles de valores de verdad
        asignaremos a la primera mitad de las filas de $p$ el valor 0, y a la
        otra mitad, el valor 1. Luego, a $q$ le asignaremos 0 a la mitad de la
        mitad de $p$, luego 1 a la otra mitad de la mitad y así sucesivamente.
        \textbf{Ejemplo:}
  \begin{table}[H]
    \centering
    \begin{tabular}{c|c}
      $p$ & $q$ \\
      \hline
      \hline
      0 & 0 \\
      0 & 1 \\
      1 & 0 \\
      1 & 1 \\
    \end{tabular}
  \end{table}
  \item Para cada fórmula, indique si corresponde a una tautología, a una
        contingencia o a una contradicción.
\end{itemize}

\begin{enumerate}

  \item $\sim (p \vee \sim p) \land (q \land(\sim p \vee r))$ \qquad
        (Contradicción)

\begin{table}[H]
  \centering
  \begin{tabular}{c|c|c||c|c|c|c|c}
    $p$ & $q$ & $r$ & $\sim p$ & \begin{tabular}[c]{@{}c@{}}$\sim (p \vee \sim p)$\\ \textcircled{\raisebox{-0.9pt}{A}}\end{tabular} & $\sim p \vee r$ & \begin{tabular}[c]{@{}c@{}}$q \land (\sim p \vee r)$\\ \textcircled{\raisebox{-0.9pt}{B}}\end{tabular} & $\textcircled{\raisebox{-0.9pt}{A}} \land \textcircled{\raisebox{-0.9pt}{B}}$ \\ \hline \hline
    0 & 0 & 0 & 1 & 0 & 1 & 0 & 0 \\ \hline
    0 & 0 & 1 & 1 & 0 & 1 & 0 & 0 \\ \hline
    0 & 1 & 0 & 1 & 0 & 1 & 1 & 0 \\ \hline
    0 & 1 & 1 & 1 & 0 & 1 & 1 & 0 \\ \hline
    1 & 0 & 0 & 0 & 0 & 0 & 0 & 0 \\ \hline
    1 & 0 & 1 & 0 & 0 & 1 & 0 & 0 \\ \hline
    1 & 1 & 0 & 0 & 0 & 0 & 0 & 0 \\ \hline
    1 & 1 & 1 & 0 & 0 & 1 & 1 & 0 \\
  \end{tabular}
\end{table}

\newpage

  \item $\sim (p \land \sim q) \leftrightarrow (p \rightarrow q)$ \qquad
        (Tautología)

\begin{table}[H]
  \centering
  \begin{tabular}{c|c|c|c|c|c}
    p & q & $\sim q$ & \begin{tabular}[c]{@{}c@{}}$\sim (p \land \sim q)$\\ \textcircled{\raisebox{-0.9pt}{A}}\end{tabular} & \begin{tabular}[c]{@{}c@{}}$p \rightarrow q$\\ \textcircled{\raisebox{-0.9pt}{B}}\end{tabular} & $\textcircled{\raisebox{-0.9pt}{A}} \leftrightarrow \textcircled{\raisebox{-0.9pt}{B}}$ \\ \hline \hline
    0 & 0 & 1 & 1 & 1 & 1 \\ \hline
    0 & 1 & 0 & 1 & 1 & 1 \\ \hline
    1 & 0 & 1 & 0 & 0 & 1 \\ \hline
    1 & 1 & 0 & 1 & 1 & 1 \\
  \end{tabular}
\end{table}


  \item
        $(p \vee q) \land ((r \leftrightarrow \sim s) \rightarrow (\sim r \land p))$
        \qquad (Contingencia)

\begin{table}[H]
  \centering
  \begin{tabular}{c|c|c|c|c|c|c|c|c|c|c}
    $p$ & $q$ & $r$ & $s$ & $\sim r$ & $\sim s$ & \begin{tabular}[c]{@{}c@{}}$p \vee q$\\ \textcircled{\raisebox{-0.9pt}{A}}\end{tabular} & \begin{tabular}[c]{@{}c@{}}$r \leftrightarrow \sim s$\\ \textcircled{\raisebox{-0.9pt}{B}}\end{tabular} & \begin{tabular}[c]{@{}c@{}}$\sim r \land p$\\ \textcircled{\raisebox{-0.9pt}{C}}\end{tabular} & $\textcircled{\raisebox{-0.9pt}{B}} \rightarrow \textcircled{\raisebox{-0.9pt}{C}}$ & $\textcircled{\raisebox{-0.9pt}{A}} \land (\textcircled{\raisebox{-0.9pt}{B}} \rightarrow \textcircled{\raisebox{-0.9pt}{C}})$ \\ \hline \hline
    0 & 0 & 0 & 0 & 1 & 1 & 0 & 0 & 0 & 1 & 0 \\ \hline
    0 & 0 & 0 & 1 & 1 & 0 & 0 & 1 & 0 & 0 & 0 \\ \hline
    0 & 0 & 1 & 0 & 0 & 1 & 0 & 1 & 0 & 0 & 0 \\ \hline
    0 & 0 & 1 & 1 & 0 & 0 & 0 & 0 & 0 & 1 & 0 \\ \hline
    0 & 1 & 0 & 0 & 1 & 1 & 1 & 0 & 0 & 1 & 1 \\ \hline
    0 & 1 & 0 & 1 & 1 & 0 & 1 & 1 & 0 & 0 & 0 \\ \hline
    0 & 1 & 1 & 0 & 0 & 1 & 1 & 1 & 0 & 0 & 0 \\ \hline
    0 & 1 & 1 & 1 & 0 & 0 & 1 & 0 & 0 & 1 & 1 \\ \hline
    1 & 0 & 0 & 0 & 1 & 1 & 1 & 0 & 1 & 1 & 1 \\ \hline
    1 & 0 & 0 & 1 & 1 & 0 & 1 & 1 & 1 & 1 & 1 \\ \hline
    1 & 0 & 1 & 0 & 0 & 1 & 1 & 1 & 0 & 0 & 0 \\ \hline
    1 & 0 & 1 & 1 & 0 & 0 & 1 & 0 & 0 & 1 & 1 \\ \hline
    1 & 1 & 0 & 0 & 1 & 1 & 1 & 0 & 1 & 1 & 1 \\ \hline
    1 & 1 & 0 & 1 & 1 & 0 & 1 & 1 & 1 & 1 & 1 \\ \hline
    1 & 1 & 1 & 0 & 0 & 1 & 1 & 1 & 0 & 0 & 0 \\ \hline
    1 & 1 & 1 & 1 & 0 & 0 & 1 & 0 & 0 & 1 & 1 \\
  \end{tabular}
\end{table}

\end{enumerate}

\newpage

\section{Deducción natural}
Demuestre los siguientes secuentes utilizando deducción natural. En cada paso,
debe indicar exactamente cuál fue la regla utilizada y sobre qué fórmulas se
aplicó.

\begin{enumerate}

  \item
        $p \land (q\rightarrow (p \rightarrow s)), p \rightarrow (q \land r) \vdash p \rightarrow s$

  \begin{align*}
    &(1) \quad p \land (q\rightarrow (p \rightarrow s))  & \quad (\text{premisa}) \\
    &(2) \quad  p \rightarrow (q \land r) & \quad (\text{premisa}) \\
    &(3) \quad  p & \quad (EC1(1)) \\
    &(4) \quad  q \rightarrow (p \rightarrow s) & \quad (EC2(1)) \\
    &(5) \quad  q \land r & \quad (EI(3, 2)) \\
    &(6) \quad  q & \quad (EC1(5)) \\
    \cline{1-2}
    & \qquad p \rightarrow s & \quad (EI(6,4))
  \end{align*}


  \item $p \rightarrow (q \rightarrow r \vee s), p, r \rightarrow t, s \rightarrow t, q\land m \vdash t$

  \begin{align*}
    (1)&\quad p \rightarrow (q \rightarrow r \vee s) & \quad (\text{premisa}) \\
    (2)&\quad p & \quad (\text{premisa}) \\
    (3)&\quad r \rightarrow t & \quad (\text{premisa}) \\
    (4)&\quad s \rightarrow t & \quad (\text{premisa}) \\
    (5)&\quad q \land m & \quad (\text{premisa}) \\
    (6)&\quad q \rightarrow (r \vee s) & \quad (EI(2, 1)) \\
    (7)&\quad q & \quad (EC1(5)) \\
    (8)&\quad r \vee s & \quad (EI(7, 6)) \\
    (9)&\quad r & \quad (supuesto) \\
    (10)& \quad t & \quad (EI(9, 3)) \\
    (11)& \quad s & \quad (supuesto) \\
    (12)& \quad t & \quad (EI(11, 4)) \\
    \cline{1-2}
       & \qquad t & \quad (ED(8, 9-10, 11-12))
  \end{align*}

  \newpage

  \item $p \land (\sim q \rightarrow \sim p) \vdash (q \land p) \vee \sim p$
        \quad (PEP 1 2025-1)

        \begin{align*}
          &(1) \quad p \land (\sim q \rightarrow \sim p)  & \quad (\text{premisa}) \\
          &(2) \quad p & \quad (EC1(1)) \\
          &(3) \quad \sim q \rightarrow \sim p & \quad (EC2(1)) \\
          &(4) \quad \sim (\sim p) & \quad (IDN(2)) \\
          &(5) \quad \sim (\sim q) & \quad (MT(4, 3)) \\
          &(6) \quad q & \quad (EDN(5)) \\
          &(7) \quad q \land p & \quad (IC(6, 2))\\
          \cline{1-2}
          & \qquad (q \land p) \vee \sim p & \quad (ID1(7))
        \end{align*}


  \item
        $((p \land q) \rightarrow (r \land s)) \land ((r \land s) \rightarrow (p \land q)), t \land (t \rightarrow s) \vdash r \rightarrow p$
        \quad (PEP 1 2025-1)

        \begin{align*}
          &(1) \quad ((p \land q) \rightarrow (r \land s)) \land ((r \land s) \rightarrow (p \land q))  & \quad (\text{premisa}) \\
          &(2) \quad t \land (t \rightarrow s) & \quad (\text{premisa}) \\
          &(3) \quad (r \land s) \rightarrow (p \land q) & \quad (EC2(1)) \\
          &(4) \quad t & \quad (EC1(2)) \\
          &(5) \quad t \rightarrow s & \quad (EC2(2)) \\
          &(6) \quad s & \quad (EI(4, 5)) \\
          &(7) \quad r & \quad (supuesto) \\
          &(8) \quad r \land s & \quad (IC(7, 6)) \\
          &(9) \quad p \land q & \quad (EI(8, 3)) \\
          &(10) \quad p & \quad (EC1(9)) \\
          \cline{1-2}
          & \qquad r \rightarrow p & \quad (II(7-10))
        \end{align*}



%  \item $(p \vee q) \vee (\sim r \rightarrow s), q \rightarrow r, \sim s \vdash p \vee r$ %\quad \textbf{(Ayudantía)}

%   \begin{align*}
%     (1)&\quad (p \vee q) \vee (\sim r \rightarrow s)  & \quad (\text{premisa}) \\
%     (2)&\quad q \rightarrow r & \quad (\text{premisa}) \\
%     (3)&\quad \sim s & \quad (\text{premisa}) \\
%     (4)&\quad p \vee q & \quad (supuesto) \\
%     (5)&\quad p & \quad (supuesto) \\
%     (6)&\quad p \vee r & \quad (ID1(5)) \\
%     (7)&\quad q & \quad (supuesto) \\
%     (8)&\quad r & \quad (EI(7, 2)) \\
%     (9)&\quad p \vee r & \quad (ID2(8)) \\
%     (10)&\quad p \vee r & \quad (ED(4, 5-6, 7-9) \\
%     (11)&\quad \sim r \rightarrow s  & \quad (\text{supuesto}) \\
%     (12)&\quad \sim(\sim r) & \quad (MT(3, 11)) \\
%     (13)&\quad r & \quad (EDN(12)) \\
%     (14)&\quad p \vee r & \quad (ID2(13)) \\
%     \cline{1-2}
%     & \qquad p \vee r & \quad (ED(1, 4-10, 11-14))
%   \end{align*}
\end{enumerate}


\end{document}
